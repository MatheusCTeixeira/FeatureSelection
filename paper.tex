\documentclass{paper}

\usepackage[a4paper]{geometry}

\usepackage[square,numbers,sort]{natbib}

\usepackage{xcolor} %% texto com diferentes cores.

\usepackage[hidelinks]{hyperref}
\usepackage{url}
\usepackage{booktabs}
\usepackage{float}
\usepackage{graphicx}
\usepackage{adjustbox}
\usepackage{tabularx}
\usepackage[portuguese]{babel}
\usepackage[fixlanguage]{babelbib}

\usepackage{algorithm}
\usepackage{algpseudocode}

\renewcommand{\it}[1]{\textit{#1}} %% Shorthand to italic.

\begin{document}

\title{Proposta de Projeto - Computação Natural}
\author{Matheus Cândido Teixeira}
\maketitle

\section{Introdução}
%%% Falar sobre Falta de exemplos negativos.
Muitos classificadores em {\itshape Machine Learning} (ML) podem ser
classificados em duas categorias: classificadores \textit{multi-class} (MCC) ou
\textit{one-class} (OCC). Os MCC, que pertencem a primeira categoria, são
utilizados quando há amostras suficientes de todas as classes na qual o método
deve ser capaz de classificar. Porém, há situações em que não é possível
acumular uma quantidade estatísticamente significativa de exemplos de todas as
classes para fazer treinamento dos algoritmos~\cite{Khalifa2016, Khan2014}. Para
situações tais como estas, o OCC pode ser mais adequado, pois ele constroi um
modelo preditivo a partir de exemplos de uma única classe.

%% Falar sobre o OCC.
O OCC diverge do MCC pelo fato de ele ser treinado para apenas uma classe.  A
classe de treinamento é denominada como positivas ou classe alvo e as demais
amostras que não possuem uma classificação (ou \textit{labels}) são denominadas
negativas ou \textit{aliens} \cite{Khan2014, Perera2019}. Atualmente, o OCC vem
sendo aplicado a muitas áreas, como visão computacional, detecção de anomalias e
processamento de imagens médicas, onde é comum haver mais amostras do
comportamento normal do que de anomalias. Essa característica torna o OCC muito
atrativo para diversas aplicações, porém um fator que afeta a performance dele é
o excesso de \textit{features} que trazem pouca informação ou redundância.

Ao construir um classificador é necessário treiná-lo com amostras das diversas
classes. A capacidade de classificação está relacionada com capacidade de
distinguir as diferenças entre os dados de cada classes. Os dados de entrada,
denominados \textit{features}, determinam qual classe a amostra
pertence. Portanto, é comum encontrar \textit{datasets} com inumeras
\textit{features} para que o classificador possa extrair algum padrão dos dados
e, assim, inferir a classe da amostra de entrada.

%%% Falar sobre o excesso de features.
Um problema associado ao excesso de \textit{features} é a presença de
\textit{features} que não contribuem para a classificação. Isso pode ocorrer de
duas maneiras: (1) algumas \textit{features} do conjunto possuem muita correlação
entre si ou (2) as \textit{features} não possem nenhuma correlação com as classes de
saída. O primeiro cenário ocorre pois uma váriavel é suficiente para descrever
uma característica e as demais se tornam redundantes e passam a afetar
negativamente a performance do classificador, pois acrescenta dados que não
trazem ganho para a predição. O segundo cenário ocorre pois os excesso de
features se comporta como ruído nos dados causando \textit{bias} e reduzindo a
performance do classificador \cite{Chandrashekar2014, Miao2016, Liu1998,
  Zheng2020}.

%%% Falar sobre a seleção de features e soluções.
Para os algoritmos \textit{multi-class} a precisão não é significativamente
afetada pelo excesso de \textit{features}, porém, para os algoritmos OCC, o
efeito pode ser significativo~\cite{Perera2019}. Para resolver esse problema, um
método utilizado é o Feature Selection (FS).  O FS busca reduzir o espaço
dimensional do conjunto de \textit{features} para um subconjunto menor e ainda
capaz de descrever os dados de entrada de modo eficiente, isso é alcançado pela
remoção de \textit{features} que trazem pouca informação relevante para a
classificação ou de \textit{features} reduntantes \cite{Chandrashekar2014}.
Alguns métodos de FS e as suas características são apresentadas na
Seção~\ref{sec_revisao}.

%%% Objetivo da pesquisa.
O FS tem sido usado como uma forma de aumentar a performance do OCC.  Ele é
usado para encontrar um subconjunto \textit{optimal} de \textit{features} para
que o OCC obtenha uma performance comparável com o dos MCCs. Diversos algoritmos
têm sido propostos, como \textit{Random Feature Selection}~(RFS),
\textit{Selecting Features from Clusters}~(SFC), \textit{Selecting the Highest
  Information Gain from Feature Clusters}~(HIC) \cite{Khalifa2016}, porém apenas
um, SFC, foi capaz de gerar um selecionar as \textit{features} de modo que o OCC
obtivesse um desempenho próximo ao dos MCC.  Portanto, o objetivo deste trabalho
é utilizar o algoritmo genético (GA) como uma heurística para encontrar
subconjunto \textit{suboptimal} de \textit{ features} que maximizem a
performance do OCC e comparar a performance obtida dele com classificadores
multi-classes. O uso do GA para FS já foi feito em outros trabalhos, como em
\cite{Chandrashekar2014}. A representação de cada indivíduo é da seguinte
maneira. Um cromossomos representa um conjunto de \textit{features}, cada locus
representa uma \textit{feature} e o gene \{0, 1\} representa a presença ou a
ausência da \textit{feature}. Desse modo, neste trabalho faz a análise do
desempenho do classificar com o FS.

%%% Falar sobre a estrutura do trabalho.
Este trabalho é dividido da seguinte maneira: A Seção~\ref{sec_trabalhos_rel}
apresenta trabalhos relacionados e aplicações do FS em OCC. Na
Seção~\ref{sec_revisao} é apresentado o estado-da-arte tanto dos algoritmos FS
como dos algoritmos OCC. Na Seção~\ref{sec_metodo} é apresentado a forma
como a pesquisa foi desenvolvida. Na Seção~\ref{sec_resultados}, os resultados
da pesquisa são apresentados. Por fim, na Seção~\ref{sec_conclusao}, os
resultados da seção anterior são discutidos, e, também, é apresentado as
conclusões obtidas nessa pesquisa.


\section{Trabalhos relacionados} \label{sec_trabalhos_rel}
\citet{Khalifa2016} utilizam OCC para detecção de pre-miRNA. Os autores
identificam que o método mais comum na literatura era utilizar TCC para a
classificação. Um problema que havia nessa aplicação era a falta de exemplos
negativos para o treinamento.  A solução utilizada na literatura era selecionar
partes aleatórias consideradas não-miRNA ou a partir de sequências geradas
aleatoriamente.  Portanto, os autores sugeriram o uso de OCC para contornar a
falta de exemplos negativos.  Além disso, eles também consideraram o impacto da
quantidade de features (mais de 1000) na performance do OCC.  Assim, eles
testram diferentes algoritmos de FS e compararam o efeito na performance da
classificação feita por algoritmos TCC (o algoritmo utilizado foi o SVM)e
OCC. Os resultados indicaram que o impacto do FS é muito mais significativo para
o OCC do que para os TCCs e que, em geral, os TCC possuem melhor desempenho do
que os OCC, porém com a escolha de um FS adequado, a performance do OCC se
aproxima dos TCC. O algoritmo FS que apresentou melhor resultado foi o {\itshape
  Selecting Features from Cluster} (SFC), onde tanto o TCC quanto o OCC
atingiram mais de 95\% de precisão.

\section{Revisão da literatura} \label{sec_revisao}
%%% Falar sobre o estado da arte.
Nesta seção é apresentado o estado-da-arte tanto para os algoritmos FS como para
os OCC. Os trabalhos são classificados em duas seções: uma para abordar os
algorimos FS e outra para os OCC.

\subsection{{\itshape Feature Selection} (FS)}
\citet{Chandrashekar2014} apresentam três tipos métodos para FS: \textit{Filter
  Methods} (FM), \textit{Wrapper Methods} (WM) e \textit{Embedded Methods} (EM).
Os FM utilizam técnicas de ranking para ordenar as variáveis para seleção.  Uma
critério de ranking é utilizado para pontuar as features e remover as que estão
abaixo de um threshold. Os métodos de ranking (MR) são aplicados antes da
classificação para remover variáveis pouco relevantes (baseado na pontuação
obtida). Uma vantagem dos MR é que eles são computacionalmente leves e evitam
\textit{overfitting}. Uma desvantagem dos MR é que o subconjunto gerado pode não
ser \textit{optimal} no sentido que pode haver features redundades entre si.  O
WM utiliza um preditor (algoritmo de classificação) como um sub-elemento do
método.  Este método seleciona um subconjunto de features e aplica um algoritmo
de classificação tendo esse subconjunto como entreada. O desempenho do
classificador é avaliado com base na performance que ele obteu para aquele
subconjunto de entrada. O objetivo é determinar qual subconjunto de features
maximiza o desempenho do classificador.  O total de subconjuntos possíveis é
$2^k$, onde $k$ é a quantidade de features presentes no dataset. Devido a essa
natureza exponencial, este é um problema NP-hard, por isso, algoritmos de busca
(sequencial ou heurística) são empregados para encontrar um subconjunto de
features de modo mais eficiente.  Por fim, o EM busca contornar o problema
NP-hard do método WM. Nesse método ao invés do preditor ser um subelemento do
FS, o FS é inserido no classificador durante a fase de treinamento. Esse método
utiliza uma estatística denominada mutual information (MI), que é calculada como
a diferença: $I(Y, X) = H(Y) - H(Y | X)$, onde $H(\cdot)$ é a medida de entropia
de Shannon e $H(\cdot| X)$ é a entropia condicional\footnote{A formulação
detalhada pode ser encontrada em \cite{Chandrashekar2014}}. O MI é empregado de
forma que ele é maximizado entre a saída, $Y$, e uma feature de entrada $X_i$ e
mínimizada entre features de entrada. A consequência disso é que são apenas
mantidas as features que introduzem o máximo de informação para a predição,
enquanto não possuem redundância entre si.

%\citet{Centner1996} apresentam uma método para a eliminação de variáveis que não
%trazem informações relevantes para a classificação. \textcolor{red}{Apresentar
%  resultados.}

\subsection{{\itshape One Class Classifiers} (OCC)}
\citet{Khan2014} apresentam uma pesquisa sobre os OCCs. Nesse trabalho, os
autores criaram uma taxonomia para o estudo de problemas OCCs. A taxonomia
proposta é dividida em três categorias: dados disponíveis para treinamento,
metodologia usada e o domínio de aplicação. A primeira categoria analisa qual a
forma dos dados utilizados para treinamento, isto é, se apenas com dados
positivos, com dados positivos e algumas poucas amostras de dados negativos (ou
até mesmo artifialmente gerados) ou com dados positivos e amostras
não-rotuladas. No que tange a metodologia de classificação, os autores
identificam duas amplas categorias: One-Class Support Vector Machines (OSVM) e
não-OSVMs. A última categoria diz respeito a aplicação que o algoritmo OCC é
aplicado: classificação de textual ou outra aplicação.

\section{Materiais e Métodos} \label{sec_metodo}
\subsection{Dataset}

A seguir é apresentado como o FS foi implementado utilizando uma variante do GA,
denominado \textit{Constraint Handling Coevolutionary Genetic Algorithm}
(CHCGA). A primeira seção apresenta os detalhes da representação dos indivíduos
e em seguida é apresentado os demais operadores genéticos, como o operador de
seleção, de mutação e de \textit{crossover}.

\subsection{Representação dos invíduos em CHCGA}
O dataset utilizado neste trabalho contém 31 \textit{features} extraídas de um
conjunto de usuários que pertencem a 3 classes distintas: \textit{humans},
\textit{bots} e \textit{news}. Portanto, para representar um indivíduo, um vetor
com 31 elementos é utilizado. Cada elemento do vetor pode assumir um valor
binário, onde 1 indica que a \textit{feature} pertence ao subconjunto de
\textit{features} utilizadas no treinamento do classificador e 0 indica a
ausência.

\subsection{Operadores evolucionários}
O primeiro operador a ser definido é o operador de \textit{fitness}, que é
responsável por mensurar a qualidade do indivíduo. Neste trabalho, a
\textit{fitness} de um indíviduo é igual a \textit{Area Under the Curve} (AUC)
da \textit{Receiver Operating Characteristic} (ROC). No CHCGA, um \textit{Highly
  Disruptive Half Uniform Crossover} (HUX) é usado como operador de
\textit{crossover}, o mecânismo desse operador consiste em selecionar dois
indivíduos na população (sem reposição) e avaliar se a distância de Hamming
entre eles é maior que um \textit{threshold} pré-definido, caso a distância
exceda esse \textit{threshold}, metade dos alelos diferentes entre eles são
alterados. Geralmente, esse \textit{threshold} é definido como $L/4$, onde $L$ é
o tamanho do indivíduo e é decrementado caso não haja decendentes suficientes.
No caso de o \textit{threshold} chegar a 0 e não houver uma nova geração
geradas, o melhor indivíduo da população é tomado como template e $N-1$
indivíduos são gerados aplicando-se o operador de mutação com probabilidade
entre 35-40\%, onde $N$ é o tamanho da população. O
\hyperref[alg:chcga]{Algoritmo~\ref*{alg:chcga}} sumariza o algoritmo.

Com esse operadores definidos, o \textit{wrapper} FS é adotado para obter o
subconjunto de \textit{features} que maximizem o valor da AUC do classificador
\textit{One-Class}. O funcionamento da estrutura é da seguinte maneira: uma
população inicial é gerada contendo $P_0$ indivíduos. Cada indivíduo possui 31
genes (cada um representando uma \textit{feature}).  Quando o valor do gene for
1, a \textit{feature} representada por esse gene é incluida no treinamento do
classificador. Para o cálculo da \textit{fitness}, $50\%$ do dataset é utilizado
como treinamento e o retante para teste. O valor da AUC resultante da predição
do classificador após o treinamento é o valor da \textit{fitness}. Portanto,
esse é um problema de maximização.

\subsection{Classificadores}
Para analisar o efeito do FS, diversos classificadores são verificados para
analisar a variação da AUC pela técnica. Os classificadores avaliados são
SVM, \textit{Logistic Regression}, \textit{Naive Bayes}, \textit{Random Forest},
\textit{AdaBoost}, KNN e ocSVM (one-class SVM).


\section{Resultados} \label{sec_resultados}
Para verificar o efeito da FS e como ela afeta a AUC dos classificadores, os
seguintes parâmetros são mantidos fixos entre os classificadores: $p_c=0.9$,
$p_m=0.2$, $Pop=100$, $Ger=60$, $\mathrm{S_{prop}=0.5}$ e $k=7$. Os resultados
indicam que o \textit{Random Forest} é o classificador que atinge o maior valor
de fitness, isto conconda com vários trabalhos na literatura que o utilizam como
classificador para o mesmo fim \cite{Bessi2016}. Entretanto, o classificador que
apresentou o maior ganho na AUC foi o AAAAA, onde houve um ganho de NN\% no
valor da AUC quando comparado com o dataset contendo todas as \it{features}. A
\hyperref[tbl:auc]{Tabela~\ref*{tbl:auc}} apresenta a média e o desvio padrão
resultante da várias iterações do FS, mantendo todos os parâmetros fixos e
alterando apenas a semente que alimenta o gerador de números aleatórios. Os
dados da tabela indicam que o classificador que atinge a maior aumento da AUC
foi o AAAAA, porém o que é mais estável (menor variância) foi o BBBBBB. A razão
para esse fato se deve ao fato de que o classificador BBBBBB utiliza $\cdots$
{\color{red} Colocar uma justificativa.}

\begin{table}[h]
  \caption{Resultado do FS para diversos classificadores}\label{tbl:auc}
  \centering
  \begin{tabular}{c | c | c | c | c}
    \toprule
    Classificador & Média & Desvio Padrão & Máximo & Mínimo \\
    \midrule
    SVM & & & & \\
    \it{Logistic Regression} & & & & \\
    \it{Naive Bayes} & & & & \\
    \it{Random Forest} & & & &\\
    \it{AdaBoost} & & & &\\
    KNN & & & &\\
    ocSVM & & & &\\
    \bottomrule
  \end{tabular}
\end{table}

Para avaliar o progresso ou convergência do FS ao longo das gerações, a Figura~1
apresenta o valor máximo (borda superior), mínimo (borda inferior) e média de
cada classificador. Como é possível verificar através da figura, a \it{fitness}
é sempre decrescente, pois a população foi gerada com o eletismo ativado, o
propósito para o uso de eletismo é para não perder boas soluções encontradas em
gerações anteriores. {\color{red} Verificar a diferença entre classificadores
  OCC e MCC}.


\section{Discussão e Conclusão} \label{sec_conclusao}

\bibliography{bibliography} \bibliographystyle{plainnat}

\end{document}
